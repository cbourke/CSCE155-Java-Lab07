\documentclass[12pt]{scrartcl}



\setlength{\parindent}{0pt}
\setlength{\parskip}{.25cm}

\usepackage{graphicx}

\usepackage{xcolor}

\definecolor{darkred}{rgb}{0.5,0,0}
\definecolor{darkgreen}{rgb}{0,0.5,0}
\usepackage{hyperref}
\hypersetup{
  letterpaper,
  colorlinks,
  linkcolor=red,
  citecolor=darkgreen,
  menucolor=darkred,
  urlcolor=blue,
  pdfpagemode=none,
  pdftitle={CS1 - Lab 7.0 - Java},
  pdfauthor={Christopher M. Bourke},
  pdfkeywords={}
}

\definecolor{MyDarkBlue}{rgb}{0,0.08,0.45}
\definecolor{MyDarkRed}{rgb}{0.45,0.08,0}
\definecolor{MyDarkGreen}{rgb}{0.08,0.45,0.08}

\definecolor{mintedBackground}{rgb}{0.95,0.95,0.95}
\definecolor{mintedInlineBackground}{rgb}{.90,.90,1}

%\usepackage{newfloat}
\usepackage[newfloat=true]{minted}
\setminted{mathescape,
               linenos,
               autogobble,
               frame=none,
               framesep=2mm,
               framerule=0.4pt,
               %label=foo,
               xleftmargin=2em,
               xrightmargin=0em,
               startinline=true,  %PHP only, allow it to omit the PHP Tags *** with this option, variables using dollar sign in comments are treated as latex math
               numbersep=10pt, %gap between line numbers and start of line
               style=default, %syntax highlighting style, default is "default"
               			    %gallery: http://help.farbox.com/pygments.html
			    	    %list available: pygmentize -L styles
               bgcolor=mintedBackground} %prevents breaking across pages
               
\setmintedinline{bgcolor={mintedBackground}}
\setminted[text]{bgcolor={mintedBackground},linenos=false,autogobble,xleftmargin=1em}
%\setminted[php]{bgcolor=mintedBackgroundPHP} %startinline=True}
\SetupFloatingEnvironment{listing}{name=Code Sample}
\SetupFloatingEnvironment{listing}{listname=List of Code Samples}

\title{CSCE 155 - Java}
\subtitle{Lab 7.0 - Arrays \& Dynamic Memory}
\author{~}
\date{~}

\begin{document}

\maketitle

\section*{Prior to Lab}

Before attending this lab:
\begin{enumerate}
  \item Read and familiarize yourself with this handout.
  \item Review the following free textbook resources:
	\begin{itemize}
  	  \item Oracle's Java Tutorial on arrays:\\
		\url{http://docs.oracle.com/javase/tutorial/java/nutsandbolts/arrays.html}
	  \item Information on Java's Garbage Collection:\\
		\url{http://www.oracle.com/technetwork/java/javase/tech/index-jsp-140228.html}
	  \item Optionally read up on Perspectives on Garbage Collection: 
		\url{http://www.oracle.com/technetwork/articles/java/garbagecollection-488837.html}
	\end{itemize}
\end{enumerate}

\section*{Peer Programming Pair-Up}

To encourage collaboration and a team environment, labs will be
structured in a \emph{pair programming} setup.  At the start of
each lab, you will be randomly paired up with another student 
(conflicts such as absences will be dealt with by the lab instructor).
One of you will be designated the \emph{driver} and the other
the \emph{navigator}.  

The navigator will be responsible for reading the instructions and
telling the driver what to do next.  The driver will be in charge of the
keyboard and workstation.  Both driver and navigator are responsible
for suggesting fixes and solutions together.  Neither the navigator
nor the driver is ``in charge.''  Beyond your immediate pairing, you
are encouraged to help and interact and with other pairs in the lab.

Each week you should alternate: if you were a driver last week, 
be a navigator next, etc.  Resolve any issues (you were both drivers
last week) within your pair.  Ask the lab instructor to resolve issues
only when you cannot come to a consensus.  

Because of the peer programming setup of labs, it is absolutely 
essential that you complete any pre-lab activities and familiarize
yourself with the handouts prior to coming to lab.  Failure to do
so will negatively impact your ability to collaborate and work with 
others which may mean that you will not be able to complete the
lab.  

\section{Lab Objectives \& Topics}
At the end of this lab you should be familiar with the following
\begin{itemize}
  \item Understand dynamic memory management and how the Java garbage collector operates
  \item Declare and fill an array with values
  \item Access an array's values individually and iterate over the entire array
  \item Pass an array as a parameter to a function
\end{itemize}

\section{Background}

Dynamic memory management involves allocating memory when 
needed and freeing up after it is no longer needed.  Poorly written 
programs that do not handle memory properly can crash or cause 
memory leaks.  A memory leak occurs when all references to a 
chunk of memory are lost, but the memory was not properly cleaned 
up.  This can lead to wasted resources and a substantial slow-down 
in performance.

Java, like many modern programming languages offers automatic 
garbage collection.  In the case of Java, the Java Virtual Machine 
manages memory for you: once an object is no longer being 
referenced by another object, it is available to be garbage collected 
and its memory freed up for further use by the JVM or by the operating 
system.  In general, you have no control over when and how garbage 
collection is performed; the JVM decides when it is best and how best 
to go about it.

A typical example in Java:

\begin{minted}{java}
String message = new String("Hello World!");
message = null;
\end{minted}

The code snippet above allocates new memory space to hold the 
\mintinline{java}{String} object containing the \mintinline{java}{"Hello World!"} 
message.  The variable message is holding a reference to that 
\mintinline{java}{String} object; it points to the memory location where 
the string is being stored.  In the second line when we reassign what 
the message variable is pointing to (to null) then all references to that 
original memory location are lost.  In many programming languages 
the string would persist in memory until the program ended.  In Java, 
it becomes available to be garbage collected and may, at some point, 
be cleaned up by the JVM.

Arrays hold collections of variable values of a certain type 
(\mintinline{java}{int} or \mintinline{java}{double} for example).
In Java, arrays can dynamically allocated using the keyword
\mintinline{java}{new}.  A few examples:

\begin{minted}{java}
int a[] = new int[100];
int n = 10;
double vector[] = new int[n];
double matrix[][] = new int[n][n];
\end{minted}

The syntax requires that an integer size be indicated when the array 
is dynamically created.  Each array is filled with the same default 
values as the type of variables it holds.  Moreover, as a convenience, 
Java keeps track of the size of arrays and gives you access to the 
size through the \mintinline{java}{length} property:

\begin{minted}{java}
for(int i=0; i<a.length; i++) { 
  System.out.println(a[i]); 
}
\end{minted}

\section{Activities}

For this lab, we'll be using several command line tools to observe memory
usage.  Therefore, we'll be building and running from the command line.
You can still use Eclipse to clone the project and edit source files but
it is suggested that you work from the command line for this lab.  Clone the repository from GitHub from the command line using the
following:

\mintinline{text}{git clone https://github.com/cbourke/CSCE155-Java-Lab07}

\subsection{Observing Garbage Collection}

In this exercise, we'll observe a memory leak in action and fix it.

\subsubsection*{Instructions}
\begin{enumerate}
  \item Open the \mintinline{text}{MemoryLeakA.java} source file and 
  	familiarize yourself with what it does.  
  \item Move to the project's root directory, \mintinline{text}{CSCE155-Java-Lab07}
  \item Compile and build the project using the Apache Ant build script 
	(\mintinline{text}{build.xml}) by executing the following on the command 
	line: \mintinline{text}{ant compile}
  \item Now run the program by executing the following from the command line:\\
	\mintinline{text}{java -cp build/classes unl.cse.memory.MemoryLeakA 10}
  \item Reopen the source file and comment out the line that prints the $i$-th 
	order statistic; recompile with the same ant command
  \item Now run the program by executing the following from the command line 
	(this will log information on when Java's garbage collector is executing 
	and how it affects memory):\\
	\mintinline[fontsize=\footnotesize]{text}{java -verbose:gc -XX:+PrintGCDetails -cp build/classes unl.cse.memory.MemoryLeakA 1000000}

	%note: we give them this directly as doing it through ant/build.xml does not display the JVM output
  \item Observe the output for a couple of minutes, but ultimately you may 
	have to kill your program (control-c)
  \item Answer the questions in your handout.
\end{enumerate}

\subsection{Breaking Garbage Collection}

Java's garbage collection relieves the user from having to worry about 
cleaning up memory, but it does not mean that we can ignore dynamic 
memory issues altogether in Java.  Java can still have ``memory leaks'' 
if poorly written code holds on to references even though it no longer 
needs them.  If objects and data continue to have valid references, they 
are not eligible for garbage collection and remain resident in memory.  
In this exercise we'll observe such a situation.

\subsubsection*{Instructions}
\begin{enumerate}
  \item Open the \mintinline{text}{MemoryLeakB.java} source file and compare 
  	that to the other demo file--what are the key differences?
  \item Execute this program in the background from the command line via the following
	(the ampersand runs the process in the background):\\
	\mintinline{text}{java -cp build/classes unl.cse.memory.MemoryLeakB 1000000 &}
  \item To monitor how much memory your program is using, start the top program:
	\mintinline{text}{top -u login} where \mintinline{text}{login} is replaced with your cse login. 
	Your Java process should be the top process.
  \item Observe the performance of your program for a couple of minutes and then 
	kill it: quit top by typing `q'  then type the command \mintinline{text}{kill 1234} 
	where \mintinline{text}{1234} is the ID of your Java process (first column in top).
  \item Answer the questions on your worksheet.
\end{enumerate}
	
\subsection{Using Arrays}

In Java, arrays can be used as parameters/arguments in methods and returned 
as values by methods.  The syntax for doing this is straightforward:

\begin{minted}{java}
public static int[] someMethod(int a, double arr[]) { 
  //code
}
\end{minted}

This defines a method that returns an integer array and takes, 
as its second argument, an array of \mintinline{java}{double} 
variables.  In this exercise, you will write and use several 
functions that utilize arrays.

\subsubsection*{Instructions}
\begin{enumerate}
  \item Open \mintinline{text}{Statistics.java} in the editor of your choice.  There 
	are three incomplete methods: \mintinline{java}{findMax}, \mintinline{java}{findMin}, 
	and \mintinline{java}{findMean}.  
  \item Fill in the missing parameters for each method.  Your goal for each function is 
  	to find their respective statistic of a list of numbers (i.e., you'll need to find the 
	mean of an array of integers in \mintinline{java}{findMean}).  That means you'll 
	have to pass in an array and the size of the array.
  \item The calls to the methods in the \mintinline{java}{main} method are also incomplete.  
	Pass each method the appropriate variables.
  \item Fill in the method bodies to find the correct statistic.  The minimum and maximum 
	are the smallest and largest numbers in the list, and the mean is the sum of the 
	list divided by its size.  
  \item Compile and test your code by using Ant as before; note you can use Ant to 
	run your program by using \mintinline{text}{ant run-stats}
  \item Complete the questions on your worksheet 
\end{enumerate}

\section{Handin/Grader Instructions}

\begin{enumerate}
  \item Hand in your completed files:
    \begin{itemize}
    \item \mintinline{text}{Statistics.java}
    \item \mintinline{text}{worksheet.md}
  \end{itemize}
  through the webhandin (\url{https://cse-apps.unl.edu/handin}) 
  using your cse login and password.  
  \item Even if you worked with a partner, you \emph{both} should
  turn in all files.
  \item Verify your program by grading yourself through the
  webgrader (\url{https://cse.unl.edu/~cse155h/grade/}) using the
  same credentials.
  \item Recall that both expected output and your program's output
  will be displayed.  The formatting may differ slightly which is fine.
  As long as your program successfully compiles, runs and outputs 
  the \emph{same values}, it is considered correct.
\end{enumerate}
	
\section{Advanced Activities (Optional)}

\subsection{Activity 1}

The Java Collections library offers alternatives to primitive arrays called lists which 
are object-oriented ordered collections that allow the user to add, retrieve, and 
remove elements from it without having to worry about the details of how elements 
are held.  An example of its usage:

\begin{minted}{java}
List<Integer> list = new ArrayList<Integer>();
list.add(10);
list.add(20);
System.out.println("second element: "+list.get(1));
\end{minted}

Read Oracle's Tutorial on the Collections framework: 
\url{http://docs.oracle.com/javase/tutorial/collections/index.html} and modify 
the exercises in this lab to utilize a list rather than arrays.

\subsection{Activity 2}
So far we've only been working with single dimension arrays, but often times it 
is important to represent something like a table or matrix in code.  One way to 
accomplish this is to use a multidimensional array.  You can declare a statically 
allocated 2D array with the statement 

\mintinline{java}{int matrix[100][50];}

You can think of 100 as the number of rows in the table and 50 as the number 
of columns.  Begin by writing a program that will initialize a matrix with some 
random numbers and find the average of each row (you'll want to use nested 
for loops).  Once you've gotten the hang of accessing the elements in a 2D
matrix, write a program that will find the product of two (square) matrices.  
To assist you, write a function that will automatically create a randomly populated 
two dimensional array and return a reference to it.

\end{document}
